%% BioMed_Central_Tex_Template_v1.06
%%                                      %
%  bmc_article.tex            ver: 1.06 %
%                                       %

%%IMPORTANT: do not delete the first line of this template
%%It must be present to enable the BMC Submission system to
%%recognise this template!!

%%%%%%%%%%%%%%%%%%%%%%%%%%%%%%%%%%%%%%%%%
%%                                     %%
%%  LaTeX template for BioMed Central  %%
%%     journal article submissions     %%
%%                                     %%
%%          <8 June 2012>              %%
%%                                     %%
%%                                     %%
%%%%%%%%%%%%%%%%%%%%%%%%%%%%%%%%%%%%%%%%%


%%%%%%%%%%%%%%%%%%%%%%%%%%%%%%%%%%%%%%%%%%%%%%%%%%%%%%%%%%%%%%%%%%%%%
%%                                                                 %%
%% For instructions on how to fill out this Tex template           %%
%% document please refer to Readme.html and the instructions for   %%
%% authors page on the biomed central website                      %%
%% http://www.biomedcentral.com/info/authors/                      %%
%%                                                                 %%
%% Please do not use \input{...} to include other tex files.       %%
%% Submit your LaTeX manuscript as one .tex document.              %%
%%                                                                 %%
%% All additional figures and files should be attached             %%
%% separately and not embedded in the \TeX\ document itself.       %%
%%                                                                 %%
%% BioMed Central currently use the MikTex distribution of         %%
%% TeX for Windows) of TeX and LaTeX.  This is available from      %%
%% http://www.miktex.org                                           %%
%%                                                                 %%
%%%%%%%%%%%%%%%%%%%%%%%%%%%%%%%%%%%%%%%%%%%%%%%%%%%%%%%%%%%%%%%%%%%%%

%%% additional documentclass options:
%  [doublespacing]
%  [linenumbers]   - put the line numbers on margins

%%% loading packages, author definitions

\documentclass[twocolumn]{bmcart}% uncomment this for twocolumn layout and comment line below
%\documentclass{bmcart}

%%% Load packages
\usepackage{amsthm,amsmath}
\usepackage{siunitx}
\usepackage{mfirstuc}
%\RequirePackage{natbib}
\usepackage[colorinlistoftodos]{todonotes}
\RequirePackage{hyperref}
\usepackage[utf8]{inputenc} %unicode support
%\usepackage[applemac]{inputenc} %applemac support if unicode package fails
%\usepackage[latin1]{inputenc} %UNIX support if unicode package fails
\usepackage[htt]{hyphenat}

\usepackage{array}
\newcolumntype{L}[1]{>{\raggedright\let\newline\\\arraybackslash\hspace{0pt}}p{#1}}

%%%%%%%%%%%%%%%%%%%%%%%%%%%%%%%%%%%%%%%%%%%%%%%%%
%%                                             %%
%%  If you wish to display your graphics for   %%
%%  your own use using includegraphic or       %%
%%  includegraphics, then comment out the      %%
%%  following two lines of code.               %%
%%  NB: These line *must* be included when     %%
%%  submitting to BMC.                         %%
%%  All figure files must be submitted as      %%
%%  separate graphics through the BMC          %%
%%  submission process, not included in the    %%
%%  submitted article.                         %%
%%                                             %%
%%%%%%%%%%%%%%%%%%%%%%%%%%%%%%%%%%%%%%%%%%%%%%%%%


%\def\includegraphic{}
%\def\includegraphics{}

%%% Put your definitions there:
\startlocaldefs
\endlocaldefs


%%% Begin ...
\begin{document}

%%% Start of article front matter
\begin{frontmatter}

\begin{fmbox}
\dochead{Report from 2015 OHBM Hackathon (HI)}

%%%%%%%%%%%%%%%%%%%%%%%%%%%%%%%%%%%%%%%%%%%%%%
%%                                          %%
%% Enter the title of your article here     %%
%%                                          %%
%%%%%%%%%%%%%%%%%%%%%%%%%%%%%%%%%%%%%%%%%%%%%%

\title{Advancing Open Science through NiData}
\vskip2ex
\projectURL{Project URL: \url{http://github.com/nidata/nidata}}

\author[
addressref={aff1},
corref={aff1},
email={bcipolli@ucsd.edu}
]{\inits{BC} \fnm{B} \snm{Cipollini}}
\author[
addressref={aff2},
%
email={arokem@uw.edu}
]{\inits{AR} \fnm{A.} \snm{Rokem}}

%%%%%%%%%%%%%%%%%%%%%%%%%%%%%%%%%%%%%%%%%%%%%%
%%                                          %%
%% Enter the authors' addresses here        %%
%%                                          %%
%% Repeat \address commands as much as      %%
%% required.                                %%
%%                                          %%
%%%%%%%%%%%%%%%%%%%%%%%%%%%%%%%%%%%%%%%%%%%%%%

\address[id=aff1]{%
  \orgname{University of California, San Diego},
  \city{La Jolla},
  \street{9500 Gilman Dr.},
  \postcode{92093},
  \postcode{California},
  \cny{USA}
}
\address[id=aff2]{%
  \orgname{University of Washington},
  \city{Seattle},
  \street{3910 15th Ave NE, Seattle},
  \postcode{98195},
  \postcode{Washington},
  \cny{USA}
}

%%%%%%%%%%%%%%%%%%%%%%%%%%%%%%%%%%%%%%%%%%%%%%
%%                                          %%
%% Enter short notes here                   %%
%%                                          %%
%% Short notes will be after addresses      %%
%% on first page.                           %%
%%                                          %%
%%%%%%%%%%%%%%%%%%%%%%%%%%%%%%%%%%%%%%%%%%%%%%

\begin{artnotes}
\end{artnotes}

%\end{fmbox}% comment this for two column layout

%%%%%%%%%%%%%%%%%%%%%%%%%%%%%%%%%%%%%%%%%%%%%%
%%                                          %%
%% The Abstract begins here                 %%
%%                                          %%
%% Please refer to the Instructions for     %%
%% authors on http://www.biomedcentral.com  %%
%% and include the section headings         %%
%% accordingly for your article type.       %%
%%                                          %%
%%%%%%%%%%%%%%%%%%%%%%%%%%%%%%%%%%%%%%%%%%%%%%

%\begin{abstractbox}

%\begin{abstract} % abstract
	
%Blank Abstract

%\end{abstract}



%%%%%%%%%%%%%%%%%%%%%%%%%%%%%%%%%%%%%%%%%%%%%%
%%                                          %%
%% The keywords begin here                  %%
%%                                          %%
%% Put each keyword in separate \kwd{}.     %%
%%                                          %%
%%%%%%%%%%%%%%%%%%%%%%%%%%%%%%%%%%%%%%%%%%%%%%

%\vskip1ex

%\projectURL{\url{http://github.com/nidata/nidata}}
%\projectURL{http://github.com/nidata/nidata}

% MSC classifications codes, if any
%\begin{keyword}[class=AMS]
%\kwd[Primary ]{}
%\kwd{}
%\kwd[; secondary ]{}
%\end{keyword}

%\end{abstractbox}
%
\end{fmbox}% uncomment this for twcolumn layout

\end{frontmatter}

%{\sffamily\bfseries\fontsize{10}{12}\selectfont Project URL: \url{http://github.com/nidata/nidata}}

%%% Import the body from pandoc formatted text
\section{Introduction}\label{introduction}

The goal of this project is to improve accessibility of open datasets by
curating them. \href{http://github.com/nidata/nidata}{``NiData''} aims
to provide a common interface for documentation, downloads, and examples
to all open neuroimaging datasets, making data usable for experts and
non-experts alike.

\section{Approach}\label{approach}

Open datasets promise to allow more thorough analysis of hard-to-collect
data and re-analysis using state-of-the-art analysis methods. However,
open datasets are not truly open unless they are easy to find, simple to
access, and have sufficient documentation for use. Currently, publicly
available data in neuroscience are scattered across a number of websites
and databases, without a common data format no common method for data
access, and varying levels of documentation. Datasets are being uploaded
to public databases through a number of initiatives, including
\href{http://www.openfmri.org/}{OpenFMRI} and
\href{http://www.nitrc.org}{NITRC}. In addition, there are funded
efforts for collecting data explicitly for the purpose of public sharing
-- most visibly in the \href{http://www.humanconnectome.org/}{Human
Connectome Project (HCP)} - but also in the
\href{http://pingstudy.ucsd.edu/}{Pediatric Imaging, Neurocognition and
Genetics (PING)} study. There are a number of funded efforts to collect
old data and re-release as public databases, notably the
\href{http://fcon_1000.projects.nitrc.org/indi/IndiRetro.html}{INDI}\cite{Mennes2013}
efforts (which include the popular ABIDE and F1000 datasets). The
\href{http://braininitiative.nih.gov/}{BRAIN initiative} aims to collect
data that will be a
\href{http://www.brainupdate.nih.gov/calling-all-statisticians/}{challenge
to store, let alone analyze}. There are even online journals focused on
publishing datasets (e.g. \href{http://www.nature.com/sdata/}{Nature
Scientific Data}), or with options to release data (e.g.
\href{http://f1000research.com/articles?tab=ALL\&articleTypes=DATA_NOTE\&subjectArea=396}{F1000
``Data Notes''}).

NiData is a Python package that provides a single interface accessing
data from a variety of open data sources. The software framework makes
it easy to add new data sources, simple to define and to provide access
to multiple datasets from a single data source. Software dependencies
are managed on a per-dataset basis, allowing downloads and examples to
use any public packages without requiring installation of packages
required by unused datasets. The interface also allows selective
download of data (by subject or type) and caches files locally, allowing
easy management of big datasets.

\section{Results}\label{results}

We focused on exposing new methods for downloading data from the HCP. We
were able to provide a downloader that accepts login credentials and
downloads files locally. We created
\href{https://github.com/arokem/nidata/blob/bcipolli-ohbm2015-ipynb/ipynb/hcp-fetcher-dwi.ipynb}{an
example} that interacts with \href{https://github.com/nipy/dipy}{dipy}
to produce diffusion imaging results on a single subject from the HCP.
We also worked at collecting common data sources, as well as individual
datasets stored at each data source, into
\href{https://github.com/nidata/nidata/wiki/Data-sources}{NiData's
``data sources'' wiki page}. We incorporated downloads, documentation,
and examples from the \href{http://github.com/nilearn/nilearn}{nilearn}
package and began discussion of making a more extensible object model.

Since the hackathon, we have created such an object model and migrated
all code to use it, and a Sphinx-based website is under development.

\section{Conclusions}\label{conclusions}

Projects like NiData improve curated data access and increases the
effectivity of big data projects with open source data.

%%%%%%%%%%%%%%%%%%%%%%%%%%%%%%%%%%%%%%%%%%%%%%
%%                                          %%
%% Backmatter begins here                   %%
%%                                          %%
%%%%%%%%%%%%%%%%%%%%%%%%%%%%%%%%%%%%%%%%%%%%%%

\begin{backmatter}

\section*{Availability of Supporting Data}
More information about this project can be found at: \url{http://github.com/nidata/nidata}.

\section*{Competing interests}
None

\section*{Author's contributions}
BC and AR wrote the software and the report.

\section*{Acknowledgements}
The authors would like to thank the organizers and attendees of the 2015
OHBM Hackathon.

  
  
%%%%%%%%%%%%%%%%%%%%%%%%%%%%%%%%%%%%%%%%%%%%%%%%%%%%%%%%%%%%%
%%                  The Bibliography                       %%
%%                                                         %%
%%  Bmc_mathpys.bst  will be used to                       %%
%%  create a .BBL file for submission.                     %%
%%  After submission of the .TEX file,                     %%
%%  you will be prompted to submit your .BBL file.         %%
%%                                                         %%
%%                                                         %%
%%  Note that the displayed Bibliography will not          %%
%%  necessarily be rendered by Latex exactly as specified  %%
%%  in the online Instructions for Authors.                %%
%%                                                         %%
%%%%%%%%%%%%%%%%%%%%%%%%%%%%%%%%%%%%%%%%%%%%%%%%%%%%%%%%%%%%%

% if your bibliography is in bibtex format, use those commands:
\bibliographystyle{bmc-mathphys} % Style BST file
\bibliography{brainhack-report} % Bibliography file (usually '*.bib' )

\end{backmatter}
\end{document}
